\documentclass[11pt,a4paper]{article}
\usepackage{fontspec}
\usepackage{geometry}
\usepackage{xcolor}
\usepackage{titlesec}
\usepackage{setspace}
\usepackage{array}
\usepackage{hyperref}

\geometry{margin=2.2cm}
\setmainfont{TeX Gyre Pagella}
\hypersetup{
  colorlinks=true,
  urlcolor=blue,
  linkcolor=black
}

\titleformat{\section}{\large\bfseries\color{blue!60!black}}{}{0em}{}
\titleformat{\subsection}{\bfseries\color{blue!50!black}}{}{0em}{}

\setstretch{1.15}

\begin{document}

\begin{center}
{\LARGE \textbf{Franche-Comté / Jura French Pronunciation Guide}}\\[6pt]
{\large A clear and consonant-driven approach for learners}\\[8pt]
\textit{Compiled for Olivier – Gentoo user and mountain French enthusiast}\\[1em]
\hrule
\end{center}

\vspace{1em}

\section*{1. Introduction}

The \textbf{Franche-Comté} and \textbf{Jura} regions of eastern France preserve a distinctive and crisp form of French. 
It shares clarity with nearby Swiss French, while keeping the natural melody of mountain speech: 
rolled or tapped \textbf{R}, bright vowels, and sharply articulated consonants.

This variant is ideal for learners seeking a grounded, intelligible accent rather than the softened nasal tones of Parisian French.

\vspace{0.5em}

\section*{2. The Spirit of Jura French}

\textbf{Key qualities:}
\begin{itemize}
  \item The tongue and lips work precisely: every consonant is heard.
  \item The \textbf{R} may be rolled gently at the front of the mouth, like in Italian or Spanish.
  \item Nasal vowels exist, but are shorter and cleaner.
  \item Speech rhythm is steady and slightly stress-timed, reflecting mountain calm.
\end{itemize}

\textbf{Goal:} clarity over speed, precision over flourish.

\vspace{0.5em}

\section*{3. The Rolled R}

\textbf{How to practice:}
\begin{enumerate}
  \item Place the tongue tip just behind your upper teeth ridge.
  \item Blow air until a light vibration occurs.
  \item Begin slowly: \textit{r, rrra, rrouge, partir, terrain.}
  \item Then blend into full sentences:
    \begin{quote}
      \textit{Très rouge ce car.}\\
      \textit{C’est parti pour la montagne !}
    \end{quote}
\end{enumerate}

\textbf{Tip:} 
If you cannot roll it yet, use a clean single tap [like Spanish “pero”] — the goal is precision, not loudness.

\vspace{0.5em}

\section*{4. Consonant Clarity}

In Jura French, \textbf{T, D, K, P, S} are crisp and never swallowed.

\textbf{Practice words:}
\begin{tabular}{>{\bfseries}m{3cm}m{9cm}}
T & \textit{très, petit, partir, tout, montagne} \\
D & \textit{doux, demander, vendredi, froid} \\
K & \textit{carte, ski, quatre, musique} \\
P & \textit{petit, propre, papier, repas} \\
S & \textit{salut, assez, soupe, suisse} \\
\end{tabular}

\vspace{0.5em}
\textbf{Tip:} let a tiny puff of air follow each stop — not exaggerated, just clean.

\vspace{0.5em}

\section*{5. Vowels and Nasals}

Vowels are pure and steady. 
Avoid gliding between them as in English or over-nasalizing as in Parisian French.

\textbf{Examples:}
\begin{quote}
\textit{été} $\rightarrow$ [e-te] not [ɛtɛ]\\
\textit{eau} $\rightarrow$ [o] not [ou]\\
\textit{neuf} $\rightarrow$ [nœf] not [nœ̃f]
\end{quote}

\textbf{Mini exercise:}
\begin{quote}
\textit{Je vais à la montagne.}\\
Say slowly, each vowel separate: [ʒə ve a la mɔ̃taɲ].
\end{quote}

\vspace{0.5em}

\section*{6. Rhythm and Flow}

Jura French prefers a calm, even rhythm:
\begin{itemize}
  \item Stress slightly the first or meaningful syllable.
  \item Pause lightly between phrases.
  \item Keep sentences melodic but grounded.
\end{itemize}

\textbf{Example comparison:}

\begin{tabular}{|m{4cm}|m{10cm}|}
\hline
\textbf{Parisian French} & \textit{C’est parti} [sɛpaʁti] – fast, soft R \\
\textbf{Jura French} & \textit{C’est parti} [sɛ parti] – clear T, rolled R \\
\hline
\textbf{Parisian} & \textit{Bonjour Madame} [bɔ̃ʒuʁ madam] – nasal, glided \\
\textbf{Jura} & \textit{Bonjour Madame} [bɔ̃ʒuʁ ma-dam] – open vowels, soft rhythm \\
\hline
\end{tabular}

\vspace{0.5em}

\section*{7. Listening and Imitation}

\textbf{Recommended sources:}
\begin{itemize}
  \item \textbf{France Bleu Franche-Comté} – local radio with authentic regional pronunciation.\\
  \url{https://www.francebleu.fr/franche-comte}
  \item \textbf{RTS (Radio Télévision Suisse)} – for similar Swiss rhythm and diction.\\
  \url{https://www.rts.ch}
  \item YouTube: search “accent franc-comtois”, “accent du Jura”, “Besançon accent”.
\end{itemize}

\textbf{Practice idea:}
Repeat one short phrase after each broadcast clip. 
Focus on consonants first, then on rhythm, then on R.

\vspace{0.5em}

\section*{8. Daily Drills}

\textbf{Morning Warm-up:}
\begin{quote}
\textit{très, rouge, partir, carte, montagne, petit, propre, suisse.}
\end{quote}

\textbf{Evening Cool-down:}
\begin{quote}
\textit{Je parle français du Jura.}\\
\textit{Il fait froid ce matin.}\\
\textit{C’est une belle journée.}
\end{quote}

\vspace{1em}

\begin{center}
\textit{Clarity is warmth. Speak like the mountains breathe — calm, precise, alive.}
\end{center}

\end{document}
