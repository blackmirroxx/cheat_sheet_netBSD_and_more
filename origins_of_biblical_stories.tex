% !TEX TS-program = xelatex
\documentclass[11pt,a4paper]{article}
\usepackage{fontspec}
\usepackage{polyglossia}
\setmainlanguage{english}
\usepackage{geometry}
\geometry{margin=2.2cm}
\usepackage{longtable}
\usepackage{array}
\usepackage{booktabs}
\usepackage{titlesec}
\usepackage{hyperref}
\usepackage{color}
\usepackage{setspace}
\usepackage{enumitem}
\setstretch{1.15}

\setmainfont{TeX Gyre Pagella}

\titleformat{\section}{\large\bfseries}{\thesection.}{0.5em}{}
\titleformat{\subsection}{\normalsize\bfseries}{\thesubsection}{0.5em}{}

\title{\textbf{Origins of Biblical Motifs and Myths}}
\author{Compiled by Olivier Mutschler \& Tux ChatGPT \& Mistral le chat}
\date{October 2025}

\begin{document}
\maketitle

\section*{Introduction}
Many biblical narratives are reinterpretations of much older stories from Mesopotamia, Egypt, Canaan, Persia, and the wider Indo-European world.  
This document lists the main motifs and their known antecedents, with historical references and a scholarly bibliography.

\section{The Great Flood (Noah’s Ark)}
\textbf{Biblical:} Genesis 6–9 — Noah builds an ark and survives a divine flood.\\
\textbf{Older sources:}
\begin{itemize}
  \item \emph{Sumerian Ziusudra Epic}\footnote{Fragment from Nippur, dated c.\ 2100 BCE. See Samuel Noah Kramer, \emph{Sumerian Mythology}, 1961.}
  \item \emph{Atrahasis Epic}\footnote{Old Babylonian tablets from Sippar, c.\ 1700 BCE. See W.\ G.\ Lambert \& A.\ R.\ Millard, \emph{Atra-ḫasīs: The Babylonian Story of the Flood}, 1969.}
  \item \emph{Epic of Gilgamesh, Tablet XI}\footnote{Standard Babylonian version, discovered at Nineveh; translation by Andrew George, \emph{The Epic of Gilgamesh}, Penguin Classics, 1999.}
\end{itemize}

\section{Divine Triads and the Trinity}
\textbf{Biblical:} Father, Son, Holy Spirit.\\
\textbf{Parallels:}
\begin{itemize}
  \item Egyptian triads — Osiris–Isis–Horus\footnote{Memphis and Abydos cults, Old Kingdom onwards. See E.\ A.\ Wallis Budge, \emph{Osiris and the Egyptian Resurrection}, 1911.}; Amun–Mut–Khonsu.
  \item Babylonian — Anu–Enlil–Ea (sky, air, water)\footnote{See Thorkild Jacobsen, \emph{The Treasures of Darkness}, 1976.}.
  \item Hindu — Brahma–Vishnu–Shiva\footnote{See Wendy Doniger, \emph{The Hindus: An Alternative History}, 2009.}.
\end{itemize}

\section{Divine Hailstorm and Battlefield Miracles}
\textbf{Biblical:} Joshua 10 — hailstones defeat the Amorites.\\
\textbf{Older analogues:}
\begin{itemize}
  \item Hittite storm-god Teshub\footnote{Depicted on Yazılıkaya reliefs, 13th century BCE. See Gary Beckman, \emph{Hittite Myths}, 1999.}.
  \item Ugaritic Baal Hadad\footnote{KTU 1.2–1.4 tablets from Ras Shamra (Ugarit), 13th century BCE. See Mark S. Smith, \emph{The Ugaritic Baal Cycle}, 1994.}.
\end{itemize}

\section{Walls of Jericho and the Power of Sound}
\textbf{Biblical:} Joshua 6 — trumpets and shouting collapse city walls.\\
\textbf{Older motifs:}
\begin{itemize}
  \item Canaanite and Sumerian myths where divine thunder destroys fortresses\footnote{See J.\ Day, “Echoes of Canaanite Myth in the Hebrew Bible,” \emph{Ugarit-Forschungen}, vol.\ 23, 1991.}.
  \item Egyptian “trumpet of victory” symbolism\footnote{Eighteenth Dynasty depictions of Thutmose III’s campaigns. See Toby Wilkinson, \emph{The Rise and Fall of Ancient Egypt}, 2010.}.
\end{itemize}

\section{Virgin Birth and Divine Conception}
\textbf{Biblical:} Mary conceives by the Holy Spirit (Matthew 1, Luke 1).\\
\textbf{Earlier versions:}
\begin{itemize}
  \item Egyptian — Isis conceives Horus from Osiris’ spirit\footnote{Pyramid Texts (Utterance 364); translated in Raymond O.\ Faulkner, \emph{The Ancient Egyptian Pyramid Texts}, 1969.}.
  \item Greek — Danaë, Semele, Leda, etc.\footnote{See Robert Graves, \emph{The Greek Myths}, 1955.}.
  \item Persian — virgin mother of the Saoshyant savior\footnote{Described in the \emph{Bundahišn} (Zoroastrian cosmology). See Mary Boyce, \emph{Zoroastrians: Their Religious Beliefs and Practices}, 1979.}.
\end{itemize}

\section{Creation by the Word}
\textbf{Biblical:} “God said, ‘Let there be light.’”\\
\textbf{Older parallels:}
\begin{itemize}
  \item Egyptian — Ptah creates by speaking names\footnote{Memphite Theology, Shabaka Stone, British Museum EA 498.}.
  \item Mesopotamian — Enki shapes world through divine decrees (\emph{me})\footnote{Sumerian myth “Enki and the World Order,” ETCSL 1.1.3.}.
  \item Vedic — universe arises from sacred sound (\emph{vāc})\footnote{\emph{Rig Veda} 10.125; translation by Ralph T.\ H.\ Griffith, 1896.}.
\end{itemize}

\section{Moses and the Basket in the River}
\textbf{Biblical:} Exodus 2 — infant Moses placed in a reed basket on the Nile.\\
\textbf{Older source:} \emph{Legend of Sargon of Akkad}\footnote{Neo-Assyrian fragment, Ashurbanipal Library, 7th century BCE copy of earlier 23rd century tale; see B.\ Foster, \emph{Before the Muses}, 2005.}.  
Identical phrasing: “My mother placed me in a basket of reeds, sealed with bitumen, and set me upon the river.”

\section{Divine Law on Stone Tablets}
\textbf{Biblical:} Ten Commandments (Exodus 20).\\
\textbf{Prototype:} \emph{Code of Hammurabi}\footnote{Stele discovered at Susa, Iran; now in the Louvre (Sb 8). Dated c.\ 1750 BCE. See Martha Roth, \emph{Law Collections from Mesopotamia and Asia Minor}, 1997.}.  
King receives laws carved in stone from sun-god Shamash.

\section{Heaven, Hell, and Afterlife Judgment}
\textbf{Biblical:} Heaven for the righteous, hell for the wicked.\\
\textbf{Precursors:}
\begin{itemize}
  \item Egyptian — weighing of the heart before Osiris\footnote{\emph{Book of the Dead}, Spell 125; Papyrus of Ani, British Museum EA 10470.}.
  \item Persian — Zoroastrian Chinvat Bridge separating good and evil souls\footnote{See Yasna 19 and the Pahlavi \emph{Dadestan-i Denig}.}.
\end{itemize}

\section{Dying and Rising God Motif}
\textbf{Biblical:} Jesus dies and rises after three days.\\
\textbf{Parallels:}
\begin{itemize}
  \item Osiris (Egypt)\footnote{Mainly in Coffin Texts and the \emph{Book of the Dead}.}, Tammuz / Dumuzi (Mesopotamia)\footnote{Sumerian poem “Dumuzi’s Dream,” ETCSL 1.4.3.}, Baal (Ugarit)\footnote{KTU 1.6–1.7 tablets.}, Adonis and Dionysus (Greece)\footnote{See Walter Burkert, \emph{Greek Religion}, 1985.}.
\end{itemize}

\newpage
\section{Summary Table}
\renewcommand{\arraystretch}{1.2}
\begin{longtable}{>{\raggedright}p{3.3cm} >{\raggedright}p{3.8cm} >{\raggedright}p{3cm} >{\raggedright}p{2cm} >{\raggedright\arraybackslash}p{4cm}}
\toprule
\textbf{Biblical Motif} & \textbf{Earliest Known Source} & \textbf{Culture of Origin} & \textbf{Approx.\ Date BCE} & \textbf{Parallel / Theme} \\
\midrule
Great Flood & \emph{Epic of Gilgamesh}, Atrahasis & Mesopotamia & 2100–1700 & Divine flood survivor \\
Divine Triad & Osiris–Isis–Horus & Egypt / Sumer & 2500–1500 & Three-in-one deity \\
Divine Hailstorm & Baal Hadad vs.\ foes & Canaanite & 1200 & Storm-god battle \\
Walls of Jericho & Divine thunder myths & Sumer–Canaan & 2000–1200 & Sound destroys walls \\
Virgin Birth & Isis–Horus, Greek myths & Egypt, Greece, Persia & 2000–500 & Miraculous conception \\
Creation by Word & Ptah theology & Egypt & 2500 & Speech as creation \\
Moses in Basket & Legend of Sargon & Akkadian & 2300 & Infant on river \\
Law Tablets & Code of Hammurabi & Babylon & 1750 & Divine stone law \\
Afterlife Judgment & Book of the Dead & Egypt & 1500 & Moral weighing \\
Resurrection & Osiris, Tammuz, Adonis & Egypt–Greece & 2500–500 & Dying–rising god \\
\bottomrule
\end{longtable}

\section*{Conclusion}
The Bible did not emerge in isolation but as part of a long continuum of Near-Eastern mythic thought.  
Through centuries of cultural exchange and reinterpretation, older symbols—floods, divine speech, virgin births, and resurrections—were reshaped into a monotheistic framework that continues to influence global spirituality today.

\vspace{1cm}
\begin{center}
\textit{“History becomes legend; legend becomes myth.” — J.\,R.\,R.\ Tolkien}
\end{center}

\newpage
\section*{Selected Bibliography}

\begin{itemize}[leftmargin=1.2cm]
  \item Beckman, Gary. \emph{Hittite Myths}. Scholars Press, 1999.
  \item Boyce, Mary. \emph{Zoroastrians: Their Religious Beliefs and Practices}. Routledge, 1979.
  \item Budge, E.\ A.\ Wallis. \emph{Osiris and the Egyptian Resurrection}. London: Philip Lee Warner, 1911.
  \item Burkert, Walter. \emph{Greek Religion}. Harvard University Press, 1985.
  \item Day, John. “Echoes of Canaanite Myth in the Hebrew Bible.” \emph{Ugarit-Forschungen}, Vol.\ 23 (1991).
  \item Doniger, Wendy. \emph{The Hindus: An Alternative History}. Penguin, 2009.
  \item Faulkner, Raymond O. \emph{The Ancient Egyptian Pyramid Texts}. Oxford University Press, 1969.
  \item Foster, Benjamin. \emph{Before the Muses: An Anthology of Akkadian Literature}. CDL Press, 2005.
  \item George, Andrew. \emph{The Epic of Gilgamesh}. Penguin Classics, 1999.
  \item Graves, Robert. \emph{The Greek Myths}. Penguin, 1955.
  \item Griffith, Ralph T.\ H. \emph{The Hymns of the Rig Veda}. E.J. Lazarus, 1896.
  \item Jacobsen, Thorkild. \emph{The Treasures of Darkness: A History of Mesopotamian Religion}. Yale University Press, 1976.
  \item Kramer, Samuel Noah. \emph{Sumerian Mythology}. Harper \& Brothers, 1961.
  \item Lambert, W.\ G., \& Millard, A.\ R. \emph{Atra-ḫasīs: The Babylonian Story of the Flood}. Oxford University Press, 1969.
  \item Roth, Martha. \emph{Law Collections from Mesopotamia and Asia Minor}. Scholars Press, 1997.
  \item Smith, Mark S. \emph{The Ugaritic Baal Cycle}. Brill, 1994.
  \item Wilkinson, Toby. \emph{The Rise and Fall of Ancient Egypt}. Random House, 2010.
\end{itemize}

\end{document}

