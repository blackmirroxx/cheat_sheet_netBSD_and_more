% this is for trees e.g. nordic white cedar or cherry tree
\documentclass{article}
\usepackage{fontspec}
\usepackage{xunicode}
\usepackage{xltxtra}
\usepackage{tikz}

% Set the main font to Liberation Sans
\setmainfont{Liberation Sans}

\title{Living Mulch and Tree Root Development}
\author{}
\date{}

\begin{document}

\maketitle

\section*{Can Living Mulch Help Tree Root Development?}
Yes, it can — if done thoughtfully. Let’s address it through your key goals:

\begin{itemize}
    \item \textbf{Promote deeper rooting}
    \item \textbf{Stimulate the rhizosphere and necromass}
    \item \textbf{Use competition to your advantage}
\end{itemize}

\subsection*{Why It Works: Rhizosphere and Necromass}
The rhizosphere is the hot zone around roots, teeming with microbes, fungi, and exudates. A living mulch does three amazing things here:

\begin{itemize}
    \item \textbf{Creates Root Competition} \\
    Encourages your cedar’s roots to dive deeper or spread laterally to access nutrients.
    \item \textbf{Feeds Soil Biology} \\
    Living roots release sugars which fuel bacteria, fungi, protozoa.
    \item \textbf{Increases Necromass} \\
    When the mulch plants die or are mowed, their roots decompose, feeding the soil web and your tree indirectly.
\end{itemize}

This effect is far superior to static “dead” mulch, which only acts passively (e.g., insulation, weed control).

\subsection*{But What About Competition?}
That’s the catch.

\begin{itemize}
    \item \textbf{Bad Competition:}
    \begin{itemize}
        \item Aggressive groundcovers that steal water or nutrients (e.g., invasive grasses)
        \item Tall or woody species that shade young trees
        \item Living mulch too close to trunk = root collar rot or fungal problems
    \end{itemize}
    \item \textbf{Good Competition:}
    \begin{itemize}
        \item Low-growing, shallow-rooted species that mostly occupy the top 5–10 cm of soil
        \item Species that fix nitrogen, feed fungi, or attract pollinators
    \end{itemize}
\end{itemize}

\subsection*{Best Living Mulch for Thuja occidentalis}

\begin{tabular}{|l|l|l|}
\hline
\textbf{Plant} & \textbf{Why it's good} & \textbf{Notes} \\ \hline
Dutch white clover & Nitrogen-fixing, low growing & Keep 10–15 cm away from trunk \\ \hline
Creeping thyme & Low mat, attracts pollinators & Needs sunlight \\ \hline
Vetch (hairy vetch) & Fixes nitrogen, suppresses weeds & Mow before seeding \\ \hline
Native mosses & Acid-tolerant, slow-growing mulch & Great for forest floor feel \\ \hline
Buckwheat & Fast-growing, flowers for insects & Chop before it gets tall \\ \hline
\end{tabular}

\subsection*{Water Management Tip}
Living mulch can cause the topsoil to dry faster if it’s dense — especially in summer.

\begin{itemize}
    \item Use your hydration pits and deep watering strategy underneath or between the mulch zones.
    \item You can also chop and drop your mulch once it gets tall — creating a self-renewing mulch layer.
\end{itemize}

\subsection*{Strategic Layout (for Forestry)}

\begin{tikzpicture}
    \draw (0,0) circle (0.5cm) node {Cedar};
    \draw (0,-1) circle (2cm) node[below, yshift=-2cm] {Living mulch ring (clover, thyme)};
    \draw (0,-3) node {15–30 cm away from stem};
\end{tikzpicture}

In between rows: taller species like buckwheat or vetch. Near trunk: low-growing, non-woody, shallow-rooted species only.

\subsection*{Summary}

\begin{tabular}{|l|l|l|}
\hline
\textbf{Metric} & \textbf{Traditional Mulch} & \textbf{Living Mulch} \\ \hline
Weed suppression & ✅ & ✅ \\ \hline
Soil moisture retention & ✅ & ✅ (depends) \\ \hline
Rhizosphere stimulation & ❌ & ✅✅✅ \\ \hline
Nutrient cycling & ❌ & ✅ (esp. N) \\ \hline
Aesthetic \& biodiversity & ❌ & ✅✅ \\ \hline
Tree competition risk & ❌ & ⚠️ (manageable) \\ \hline
\end{tabular}

\end{document}

