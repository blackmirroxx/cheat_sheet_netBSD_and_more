
\documentclass[10pt]{article}
\usepackage[margin=0.75in]{geometry}
\usepackage{fontspec}
\usepackage{xcolor}
\usepackage{multicol}
\usepackage{parskip}
\usepackage{graphicx}
\usepackage{booktabs}

% Fonts
\setmainfont{DejaVu Sans}
\newfontfamily\headingfont{DejaVu Sans Bold}

% Colors
\definecolor{sectionblue}{RGB}{60, 120, 180}
\definecolor{highlightgreen}{RGB}{50, 160, 80}

% Section Style
\newcommand{\sectiontitle}[1]{\textcolor{sectionblue}{\headingfont #1}}

\begin{document}

\begin{center}
    {\LARGE \headingfont Watering Strategy for Root Growth 🌊}\\[0.5em]
    \small For 50–60 cm Nordic White Cedar Trees in Forest Soils
\end{center}

\sectiontitle{Challenge: Hydrophobic Soil}
\begin{itemize}
    \item Dry, organic-rich soil can become water-repellent.
    \item Water beads on top instead of soaking into the root zone.
    \item Young cedars with shallow roots are vulnerable.
\end{itemize}

\sectiontitle{Recommended Watering Routine}
\begin{itemize}
    \item \textbf{Deep Soak:} Once per week, apply \textbf{10–15 liters} of water slowly per tree.
    \item \textbf{Top Sprinkle:} Every 2–3 days, add \textbf{0.5 liter} to prevent surface crust.
\end{itemize}

\sectiontitle{How to Apply Water Effectively}
\begin{enumerate}
    \item Create a \textbf{20–30 cm hydration basin} around the tree.
    \item Pour water slowly to avoid runoff.
    \item Let water soak in deeply to promote root descent.
\end{enumerate}

\sectiontitle{Optional Enhancements}
\begin{itemize}
    \item Add \textbf{1 drop of unscented dish soap per liter} to break surface tension.
    \item Use \textbf{coarse mulch} (e.g., wood chips) to retain moisture, but keep 5 cm away from stem.
\end{itemize}

\sectiontitle{Summary Strategy}
\begin{itemize}
    \item Deep water once per week trains deep roots.
    \item Light surface watering prevents hydrophobic topsoil.
    \item Adjust based on weather and tree age (more deep water in drought).
\end{itemize}

\end{document}
