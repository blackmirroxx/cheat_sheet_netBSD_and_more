% !TEX TS-program = xelatex
\documentclass[11pt,a4paper]{article}
\usepackage{fontspec}
\usepackage{polyglossia}
\setmainlanguage{english}
\usepackage{geometry}
\geometry{margin=1.0cm}
\setlength{\footskip}{8mm} % default ~12mm -> smaller -> move upwards
\usepackage{longtable}
\usepackage{array}
\usepackage{booktabs}
\usepackage{titlesec}
\usepackage{hyperref}
\usepackage{setspace}
\usepackage{enumitem}
\usepackage{tikz}
\usetikzlibrary{arrows.meta, positioning, shapes.geometric}
\setstretch{1.15}
\setmainfont{TeX Gyre Pagella}

\titleformat{\section}{\large\bfseries}{\thesection.}{0.5em}{}
\titleformat{\subsection}{\normalsize\bfseries}{\thesubsection}{0.5em}{}

\title{\textbf{Origins of Qur’ānic Narratives and Their Ancient Predecessors}}
\author{Compiled by Olivier Mutschler \& Tux (ChatGPT) \& Mistral le chat}
\date{October 2025}

\begin{document}
\maketitle

\section*{Introduction}
Although the Qur’ān presents itself as divine revelation, many of its stories have demonstrable antecedents in older Jewish, Christian, Zoroastrian, and Arabian oral traditions.  
This document lists those narratives that appear “unique” to Islam and traces their possible mythological or literary predecessors.

% --- Core sections omitted for brevity (same as previous version) ---
% Copy full narrative sections from previous version here if you want to keep everything complete.
% Below we include only the ending, timeline, and bibliography to avoid redundancy.

\section{The Tribes of ʿĀd and Thamūd}
\textbf{Qur’ān:} Sūras 7:65–78; 11:50–68.\\
\textbf{Summary:} Two ancient Arabian tribes destroyed by a screaming wind and an earthquake for their arrogance.\\
\textbf{Antecedents:}
\begin{itemize}
  \item South-Arabian inscriptions (Sabaean)\footnote{Inscriptions from Marib and Sirwah (1st c.\ BCE–3rd c.\ CE) refer to divine punishment of rebellious tribes. See Christian Robin, \emph{Les Hautes-Terres du Nord-Yémen}, 1982.}
  \item Akkadian myth \emph{The Curse of Akkad}\footnote{c.\ 2100 BCE; Enlil sends drought and wind to punish Naram-Sin. See B.\ Foster, \emph{Before the Muses}, 2005.}
\end{itemize}

\section{The Companions of the Cave (Aṣḥāb al-Kahf)}
\textbf{Qur’ān:} Sūra 18:9–26.\\
\textbf{Summary:} Pious youths sleep in a cave for centuries and awaken to find the world transformed.\\
\textbf{Predecessor:}
\begin{itemize}
  \item Christian legend of the Seven Sleepers of Ephesus\footnote{Greek Acts, c.\ 250–300 CE; Syriac version by Jacob of Serugh. See B.\ Roggema, \emph{The Legend of the Seven Sleepers}, 2008.}
\end{itemize}

\section{Abraham Cast into the Fire}
\textbf{Qur’ān:} Sūra 21:51–70.\\
\textbf{Summary:} Abraham destroys idols, is thrown into a furnace, but emerges unharmed.\\
\textbf{Predecessor:}
\begin{itemize}
  \item Jewish \emph{Midrash Rabbah on Genesis}\footnote{3rd–5th c.\ CE. Nimrod throws Abraham into fire. See Jacob Neusner, \emph{Genesis Rabbah: Translation and Commentary}, 1985.}
\end{itemize}

\section{Mary and the Infant Speaking}
\textbf{Qur’ān:} Sūra 19:16–34.\\
\textbf{Summary:} Mary gives birth alone; the newborn Jesus speaks in the cradle defending her honor.\\
\textbf{Predecessor:}
\begin{itemize}
  \item \emph{Infancy Gospel of Thomas} and \emph{Arabic Gospel of the Infancy}\footnote{2nd–5th c.\ CE apocrypha. See J.\ K.\ Elliott, \emph{The Apocryphal New Testament}, 1993.}
\end{itemize}

\section{The People of the Elephant (Sūrat al-Fīl)}
\textbf{Qur’ān:} Sūra 105.\\
\textbf{Summary:} God destroys an army of elephants with birds pelting stones.\\
\textbf{Possible antecedents:}
\begin{itemize}
  \item Historical Aksumite invasion of Mecca (c.\ 570 CE)\footnote{Referenced in South-Arabian and Ethiopian chronicles; see Irfan Shahîd, \emph{Byzantium and the Arabs in the Sixth Century}, 1995.}
  \item Ancient Near-Eastern “avian plague” omens\footnote{Assyrian omen texts describing birds as bearers of divine wrath. See A.\ Livingstone, \emph{Court Poetry and Literary Miscellanea}, 1989.}
\end{itemize}

\section{The Resurrection Trumpet}
\textbf{Qur’ān:} Sūra 36:51–53.\\
\textbf{Summary:} A trumpet blast awakens the dead for judgment.\\
\textbf{Antecedents:}
\begin{itemize}
  \item New Testament — 1 Thessalonians 4:16\footnote{“The trumpet of God will sound, and the dead in Christ will rise first.”}
  \item Zoroastrian \emph{Frashokereti}\footnote{Saoshyant raises the dead with a trumpet at world’s renewal. See Mary Boyce, \emph{Zoroastrians}, 1979.}
\end{itemize}

\section{Solomon and the Jinn}
\textbf{Qur’ān:} Sūras 27:15–44; 34:12–14.\\
\textbf{Summary:} Solomon commands demons, wind, and animals through divine power.\\
\textbf{Predecessor:}
\begin{itemize}
  \item \emph{Testament of Solomon}\footnote{Jewish magical text, 1st–3rd c.\ CE. See D.\ C.\ Duling, “Testament of Solomon,” in \emph{The Old Testament Pseudepigrapha}, ed.\ J.\ H.\ Charlesworth, 1983.}
\end{itemize}

\section{Dhul-Qarnayn and the Wall Against Gog \& Magog}
\textbf{Qur’ān:} Sūra 18:83–98.\\
\textbf{Summary:} The “Two-Horned One” travels to the ends of the earth and builds a barrier to contain barbarian tribes.\\
\textbf{Predecessor:}
\begin{itemize}
  \item \emph{Alexander Romance}\footnote{Greek original 3rd c.\ BCE; Syriac version 6th c.\ CE. See R.\ Stoneman, \emph{The Greek Alexander Romance}, 1991.}
\end{itemize}

\section{Iblīs Refusing to Bow to Adam}
\textbf{Qur’ān:} Sūras 2:34; 7:11–18.\\
\textbf{Summary:} Iblīs refuses to bow before Adam and is cast out.\\
\textbf{Antecedents:}
\begin{itemize}
  \item \emph{Life of Adam and Eve}\footnote{Jewish apocryphon, 1st c.\ CE. See M.\ D.\ Johnson, \emph{The Life of Adam and Eve}, 1985.}
  \item Zoroastrian dualism\footnote{Ahriman’s rebellion against Ahura Mazda’s creation; see Zaehner, \emph{The Dawn and Twilight of Zoroastrianism}, 1961.}
\end{itemize}

\section{Harut and Marut: Angels Teaching Magic}
\textbf{Qur’ān:} Sūra 2:102.\\
\textbf{Summary:} Two angels in Babylon teach sorcery as a trial for humankind.\\
\textbf{Predecessor:}
\begin{itemize}
  \item \emph{Book of Enoch (1 Enoch 6–7)}\footnote{Watchers descend, teach forbidden arts, are punished. See R.\ H.\ Charles, \emph{The Book of Enoch}, 1912.}
\end{itemize}

\section{The Throne of Bilqīs (Queen of Sheba)}
\textbf{Qur’ān:} Sūra 27:38–44.\\
\textbf{Summary:} A jinn brings the Queen’s throne instantly to Solomon’s court.\\
\textbf{Possible antecedents:}
\begin{itemize}
  \item Persian and Indian tales of wind-spirits\footnote{Comparable to Iranian \emph{divs} and Indian \emph{vāta} legends. See J.\ Duchesne-Guillemin, \emph{La Religion de l’Iran Ancien}, 1962.}
  \item \emph{Testament of Solomon}\footnote{Demons move objects at the king’s command.}
\end{itemize}

\section{The Scale of Deeds}
\textbf{Qur’ān:} Sūras 21:47; 99:6–8.\\
\textbf{Summary:} Each soul’s deeds are weighed on a balance at judgment.\\
\textbf{Antecedents:}
\begin{itemize}
  \item Egyptian \emph{Book of the Dead}\footnote{Spell 125, “Weighing of the Heart.”}
  \item Zoroastrian \emph{Dadestan-i Denig}\footnote{Souls weighed at the Chinvat Bridge. See E.\ W.\ West, \emph{Sacred Books of the East, vol.\ 18}, 1880.}
\end{itemize}

\newpage
\section{Summary Table}
\renewcommand{\arraystretch}{1.2}
\begin{longtable}{>{\raggedright}p{3.3cm} >{\raggedright}p{3.8cm} >{\raggedright}p{3cm} >{\raggedright}p{2cm} >{\raggedright\arraybackslash}p{4cm}}
\toprule
\textbf{Qur’ānic Story} & \textbf{Closest Pre-Islamic Source} & \textbf{Cultural Origin} & \textbf{Date BCE/CE} & \textbf{Motif} \\
\midrule
ʿĀd \& Thamūd & Curse of Akkad; Arabian lore & Mesopotamian / Sabaean & 2100 BCE–3rd CE & Divine wind punishment \\
Companions of the Cave & Seven Sleepers of Ephesus & Greek–Syriac Christian & 3rd CE & Centuries-long sleep \\
Abraham in the Fire & Midrash Rabbah & Rabbinic Judaism & 3rd–5th CE & Prophet survives fire \\
Infant Jesus Speaks & Infancy Gospel of Thomas & Christian Apocrypha & 2nd CE & Speaking infant miracle \\
People of the Elephant & Aksumite invasion legend & Arabian / Ethiopian & 6th CE & Birds destroy army \\
Resurrection Trumpet & Zoroastrian Frashokereti & Persian / Christian & 1st–6th CE & Trumpet awakens dead \\
Solomon \& Jinn & Testament of Solomon & Jewish magical & 1st–3rd CE & King commands spirits \\
Dhul-Qarnayn & Alexander Romance & Hellenistic / Syriac & 3rd BCE–6th CE & Wall vs.\ Gog \& Magog \\
Iblīs and Adam & Life of Adam and Eve & Jewish / Persian dualist & 1st CE & Rebellion of angel \\
Harut \& Marut & Book of Enoch & Jewish Apocrypha & 2nd BCE & Angels teach sorcery \\
Throne of Bilqīs & Persian/Indian folklore & Iranian / Indic & pre-7th CE & Instant transport miracle \\
Scale of Deeds & Book of the Dead & Egyptian / Persian & 1500 BCE+ & Weighing of souls \\
\bottomrule
\end{longtable}

\section*{Conclusion}
The Qur’ānic corpus absorbed and reinterpreted a vast network of earlier Jewish, Christian, Persian, and Arabian traditions.  
While the stylistic expression in Arabic was innovative, the mythic building blocks—angelic descent, miraculous speech, weighing of souls, and cosmic walls—belong to a long continuum of Near-Eastern religious imagination.

\vspace{1cm}
\begin{center}
\textit{“The ink of history and the breath of myth are often drawn from the same well.”}
\end{center}

\newpage
\section*{Cultural Transmission Timeline}

\begin{center}
\begin{tikzpicture}[
  node distance=0.4cm and 0.5cm,
  every node/.style={font=\tiny, align=center},
  process/.style={rectangle, rounded corners, draw=black, fill=gray!10, minimum width=2.8cm, minimum height=1.0cm},
  arrow/.style={-{Stealth[length=3mm]}, thick}
]

\node[process, fill=orange!15] (meso) {Mesopotamia\\(Sumer, Akkad)\\\textit{2100--1700 BCE}\\Floods, divine law, wind gods};
\node[process, fill=yellow!20, right=of meso] (egypt) {Egypt\\\textit{2000--1500 BCE}\\Afterlife, judgment scales};
\node[process, fill=green!20, right=of egypt] (persia) {Persia\\(Zoroastrianism)\\\textit{1000--300 BCE}\\Dualism, resurrection};
\node[process, fill=cyan!15, right=of persia] (levant) {Levant / Israel\\\textit{900--100 BCE}\\Biblical synthesis};
\node[process, fill=purple!15, right=of levant] (christ) {Hellenistic / Christian\\\textit{300 BCE--600 CE}\\Apocrypha, Gospels};
\node[process, fill=red!20, right=of christ] (arabia) {Arabia\\\textit{6th--7th CE}\\Qur’ānic revelation};

\draw[arrow] (meso) -- (egypt);
\draw[arrow] (egypt) -- (persia);
\draw[arrow] (persia) -- (levant);
\draw[arrow] (levant) -- (christ);
\draw[arrow] (christ) -- (arabia);

\node[below=1.2cm of levant, align=center, font=\small\itshape, text width=10cm] 
{Ideas flowed across trade, empire, and scripture: \\
floods, creation by word, resurrection, moral judgment, and angelic rebellion \\
evolved from mythic cosmology into monotheistic theology.};

\end{tikzpicture}
\end{center}

\section*{Selected Bibliography}
\begin{itemize}[leftmargin=1.2cm]
  \item Boyce, Mary. \emph{Zoroastrians: Their Religious Beliefs and Practices}. Routledge, 1979.
  \item Charles, R.\ H. \emph{The Book of Enoch}. Oxford, 1912.
  \item Duling, Dennis C. “Testament of Solomon.” In \emph{The Old Testament Pseudepigrapha}, ed.\ J.\ H.\ Charlesworth, Doubleday, 1983.
  \item Duchesne-Guillemin, Jacques. \emph{La Religion de l’Iran Ancien}. Paris: PUF, 1962.
  \item Elliott, J.\ K. \emph{The Apocryphal New Testament}. Oxford University Press, 1993.
  \item Foster, Benjamin R. \emph{Before the Muses}. CDL Press, 2005.
  \item Johnson, M.\ D. \emph{The Life of Adam and Eve}. In \emph{The Old Testament Pseudepigrapha}, vol.\ 2, Doubleday, 1985.
  \item Neusner, Jacob. \emph{Genesis Rabbah: Translation and Commentary}. University Press of America, 1985.
  \item Robin, Christian. \emph{Les Hautes-Terres du Nord-Yémen}. CNRS, 1982.
  \item Roggema, Barbara. \emph{The Legend of the Seven Sleepers}. Brill, 2008.
  \item Shahîd, Irfan. \emph{Byzantium and the Arabs in the Sixth Century}. Dumbarton Oaks, 1995.
  \item Stoneman, Richard. \emph{The Greek Alexander Romance}. Penguin, 1991.
  \item West, E.\ W. (trans.) \emph{Sacred Books of the East, Vol.\ 18: Pahlavi Texts}. Oxford, 1880.
  \item Zaehner, R.\ C. \emph{The Dawn and Twilight of Zoroastrianism}. Weidenfeld \& Nicolson, 1961.
\end{itemize}

\vfill
\begin{center}
\textit{“Through empires and tongues, myth traveled farther than armies —\\
and became the scripture of civilizations.”}
\end{center}

\end{document}

